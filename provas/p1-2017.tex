\documentclass[a4paper,10pt]{article}
\usepackage[utf8]{inputenc}
\usepackage{amsmath}% $ \implies $
\usepackage{amssymb}% \blacksquare

%opening
\title{Primeira Prova de CI311}
\author{}
\date{06/11/2017}

\begin{document}

\maketitle

%\begin{abstract}

%\end{abstract}

1 (21 Pontos) Expressar os seguintes fatos como lógica proposicional:
\begin{itemize}
 \item Uma criança não é um jovem.
 \item Se um adulto é trabalhador, então ele não está aposentado.
 \item Para ser aposentado, a pessoa deve ser um adulto ou um idoso.
\end{itemize}

2 (21 Pontos) Classificar as fórmulas a seguir de acordo com sua satisfatividade, validade, falseabilidade ou insatisfabilidade.
\begin{itemize}
 \item $ (p \implies q) \implies (q \implies p) $
 \item $ p \implies \sim \sim p $
 \item $ ((p \implies q) \wedge (r \implies q)) \implies (p \vee r \implies q) $
\end{itemize}

3 (28 pontos) Colocar as fórmulas abaixo em FNC e dizer quais delas, após transformação, são cláusulas de Horn:

$ ((p \implies q) \implies p ) \implies p $

$ (p \vee q) \implies \sim (q \vee r) $

4 (30 pontos) Considere uma lógica modal cuja relação de acessibilidade, no contexto semântico, é reflexiva e transitiva. Prove que os seguintes esquemas de axioma são válidos:
\begin{itemize}
 \item $ T. \square A \implies A $
 \item $ 4. \square A \implies \square \square A $
\end{itemize}

\end{document}
