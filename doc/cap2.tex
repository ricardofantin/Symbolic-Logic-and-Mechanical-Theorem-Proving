\section{The Propositional Logic}
3. Complete the following truth table (Table 2.11 -- Tabela~\ref{tab:pergunta2.3}) of the formula

\begin{equation}
 (\sim P \vee Q) \wedge (\sim(P \vee \sim Q))
\end{equation}

\begin{table}
 \begin{tabular}{cccccccc}
  \hline
  P & Q & $\sim P$ & $\sim Q$ & $\sim P \vee Q$ & $P \wedge \vee Q$ & $\sim(P \wedge \sim Q)$ & $(\sim P \vee Q) \wedge (\sim(P \vee \sim Q))$ \\
  \hline
  T & T & & & & & & \\
  T & F & & & & & & \\
  F & T & & & & & & \\
  F & F & & & & & & \\
  \hline
 \end{tabular}
 \caption{Tabela da pergunta 3 do capítulo 2.}
 \label{tab:pergunta2.3}
\end{table}

Resposta na Tabela~\ref{tab:resposta2.3}.

\begin{table}
 \begin{tabular}{cccccccc}
  \hline
  P & Q & $\sim P$ & $\sim Q$ & $\sim P \vee Q$ & $P \wedge \sim Q$ & $\sim(P \wedge \sim Q)$ & $(\sim P \vee Q) \wedge (\sim(P \wedge \sim Q))$ \\
  \hline
  T & T & F & F & T & F & T & T \\
  T & F & F & T & F & T & F & F \\
  F & T & T & F & T & F & T & T \\
  F & F & T & T & T & F & T & T \\
  \hline
 \end{tabular}
 \caption{Tabela da resposta 3 do capítulo 2.}
 \label{tab:resposta2.3}
\end{table}

4. For each of the following formulas, determine whether it is valid, invalid, inconsistent, consistent, or some combination of these.
\begin{enumerate}
 \item[(a)] $ \sim(\sim P) \implies P $ \newline
válido, consistente
 \item[(b)] $ P \implies (P \wedge Q) $ \newline
inválido, consistente
 \item[(c)] $ \sim(P \vee Q) \vee \sim Q $ \newline
inválido, consistente
 \item[(d)] $ (P \vee Q) \implies P $ \newline
inválido, consistente
 \item[(e)] $ (P \implies Q) \implies ( \sim Q \implies \sim P) $ \newline
inválido, consistente
 \item[(f)] $ (P \implies Q) \implies ( Q \implies P) $ \newline
inválido, consistente
 \item[(g)] $ P \vee (P \implies Q) $ \newline
válido, consistente
 \item[(h)] $ (P \wedge (Q \implies P)) \implies P $ \newline
válido, consistente
 \item[(i)] $ P \vee (Q \implies \sim P) $ \newline
válido, consistente
 \item[(j)] $ (P \vee \sim Q) \wedge (\sim P \vee Q) $ \newline
inválido, consistente
 \item[(k)] $ (\sim P \wedge (\sim (P \implies Q)) $ \newline
inválido, consistente
 \item[(l)] $ P \implies \sim P $ \newline
inválido, consistente
 \item[(m)] $ \sim P \implies P $ \newline
inválido, consistente
\end{enumerate}

6. Transform the following into disjunctive normal forms.

\begin{enumerate}
 \item[(a)] $ (\sim P \wedge Q) \implies R $ \newline
$P \vee \sim Q \vee R$
 \item[(b)] $ P \implies \left( ( Q \wedge R) \implies S \right)$ \newline
$ \sim P \vee \sim Q \vee \sim R \vee S $
 \item[(c)] $ \sim (P \vee \sim Q) \wedge (S \implies T) $ \newline
$ (P \wedge \sim Q) \vee (\sim P \wedge Q) $
 \item[(d)] $ (P \implies Q) \implies R $ \newline
$ P \wedge \sim Q \vee R $
 \item[(e)] $ \sim (P \wedge Q) \wedge (P \vee Q) $. \newline
$ (\sim P \wedge \sim Q \wedge \sim S) \vee (\sim P \wedge \sim Q \wedge T) $
\end{enumerate}

7. Transform the following into conjunctive normal forms
\begin{itemize}
 \item[(a)] $ P \vee (\sim P \wedge Q \wedge R) $ \newline
$(P \vee Q) \wedge (P \vee R) $
 \item[(b)] $ \sim (P \implies Q) \vee (P \vee Q) $ \newline
$ P \vee Q $
 \item[(c)] $ \sim (P \implies Q) $ \newline
$ P \wedge \sim Q $
 \item[(d)] $ (P \implies Q) \implies R $ \newline
$ (P \vee R) \wedge (\sim Q \vee R) $
 \item[(e)] $ (\sim P \wedge Q) \vee (P \wedge \sim Q) $ \newline
$ (P \vee Q) \wedge (\sim P \vee \sim Q) $
\end{itemize}

8. Is it possible to have a formula that is in conjunctive normal form as well as disjunctive normal form? If yes, give an example.

Sim, possíveis fórmulas: P, $ P \vee Q $ ou $ P \wedge Q $.

\underline{9.} Verify each of the following pairs of equivalent formulas by transforming formulas on both sides of the sigh = into the same normal form.
\begin{itemize}
 \item[(a)] $ P \wedge P = P $, and $ P \vee P = P $ \newline
P = P e P = P
 \item[(b)] $ (P \implies Q) \wedge (P \implies R) = (P \implies (Q \wedge R)) $ \newline
$ (\sim P \vee Q) \wedge (\sim P \vee R) $
 \item[(c)] $ (P \implies Q) \implies (P \wedge Q) = (\sim P \implies Q) \wedge (Q \implies P) $ \newline
As fórmulas não representam o mesmo valor verdade. A primeira fórmula é igual a P e a segunda é $ \sim Q \vee P $.
 \item[(d)] $ P \wedge Q \wedge (\sim P \vee \sim Q) = \sim P \wedge \sim Q \wedge (P \vee Q) $ \newline
$ \square $
 \item[(e)] $ P \vee (P \implies (P \wedge Q)) = \sim P \vee \sim Q \vee (P \wedge Q) $. \newline
$ \blacksquare $
\end{itemize}

10. Prove that $ (\sim Q \implies \sim P) $ is logical consequence of $(P \implies Q) $ .

$P \implies Q$

$\sim P \vee Q$

$Q \vee \sim P$

$\sim Q \implies \sim P$

14. Show that Q is a logical consequence of $ (P \implies Q) $ and P. This is related to the so-called modus ponens rule.

Dado que P tem valor 1, o  valor de $ (P \implies Q) $ equivale ao valor de Q. Desta forma P e $ (P \implies Q) $ equivale a Q.
