\section{The First-Order Logic}
*1. Let P(x) and Q(x) represent ``x is a rational number 11'' and ``x is a real number'', respectively. Symbolize the following sentences:
\begin{itemize}
 \item[1.1] Every rational number is a real number. \newline
$ (\forall x) P(x) \implies Q(x) $
 \item[1.2] Some real numbers are rational numbers. \newline
$ (\exists x) Q(x) \implies P(x) $ \newline
ou \newline
$ (\exists x) Q(x) \wedge P(x) $
 \item[1.3] Not every real number is a rational number. \newline
$ (\exists x) Q(x) \wedge P(x) $
\end{itemize}

5. For the following interpretation (D = \{a,b\}), (Tabela~\ref{tab:pergunta3.5})
\begin{table}[h]
 \centering
 \begin{tabular}{cccc}
  \hline
  P(a,a) & P(a,b) & P(b,a) & P(b,b) \\ \hline
  T & F & F & T \\ \hline
 \end{tabular}
 \label{tab:pergunta3.5}
 \caption{Interpretação proposta no enunciado da questão 3.5.}
\end{table}
, determine the truth value of the following formulas:
\begin{itemize}
 \item[(a)] $ (\forall x) (\exists y) P(x,y) $ \newline
$ \blacksquare $
 \item[(b)] $ (\forall x) (\forall y) P(x,y) $ \newline
$ \square $
 \item[(c)] $ (\exists x) (\forall y) P(x,y) $ \newline
$ \square $
 \item[(d)] $ (\exists y) \sim P(a,y) $ \newline
$ \blacksquare $
 \item[(e)] $ (\forall x) (\forall y) (P(x,y) \implies P(y,x)) $ \newline
$ \blacksquare $
 \item[(f)] $ (\forall x) P(x,x) $ \newline
$ \blacksquare $
\end{itemize}

\underline{6.} Consider the following formula:

$ A: (\exists x) P(x) \implies (\forall x) P(x). $

a. Prove that this formula is always true if the domain D contains only one element.

Com apenas um elemento no domínio, por exemplo a, só existe um valor verdade para P(a). Sendo F ou T, $ F \implies F $ ou $ T \implies T $ tem como valor verdade T.

b. Let D = \{a,b\}. Find an interpretation over D in which A is evaluated to F.

Sendo P(a) = T e P(b) = F, a fórmula $ P(a) \implies P(b) $ resulta em F.

7. Consider the following interpretation:

Domain: D = \{1,2\}.

Assignment of constants a and b (Tabela~\ref{tab:pergunta7a}):

\begin{table}[h]
 \centering
 \begin{tabular}{cc}
  \hline
  a & b \\ \hline
  1 & 2 \\ \hline
 \end{tabular}
 \caption{Valores das constantes a e b pro exercício 7.}
 \label{tab:pergunta7a}
\end{table}

Assignment for function f (Tabela~\ref{tab:pergunta7b}):

\begin{table}[h]
 \centering
 \begin{tabular}{cc}
  \hline
  f(1) & f(2) \\ \hline
  2 & 1 \\ \hline
 \end{tabular}
 \caption{Valores da função \texttt{f()} pro exercício 7.}
 \label{tab:pergunta7b}
\end{table}

Assignment for predicate P (Tabela~\ref{tab:pergunta7c}):

\begin{table}[h]
 \centering
 \begin{tabular}{cccc}
  \hline
  P(1,1) & P(1,2) & P(2,1) & P(2,2) \\ \hline
  T & T & F & F \\ \hline
 \end{tabular}
 \caption{Valores do predicado \texttt{P()} pro exercício 7.}
 \label{tab:pergunta7c}
\end{table}

Evaluate the truth value of the following formulas in the above interpretation:
\begin{itemize}
 \item[(1)] $ P(a, f(a)) \wedge P(b, f(b)) $ \newline
$ \square $
 \item[(2)] $ (\forall x) (\exists y) P(y,x) $ \newline
$ \blacksquare $
 \item[(3)] $ (\forall x) (\forall y) (P(x,y) \implies P(f(x),f(y))) $. \newline
$ \blacksquare $
\end{itemize}

8. Let F1 and F2 be as follows:

$ F_1: (\forall x) (P(x) \implies Q(x)) $

$ F_2: \sim Q(a). $

Prove that $\sim P(a)$ is a logical consequence of F1 and F2.

De $F_1$ com $ S = \{ a \diagup x\} $ obtêm-se $ P(a) \implies Q(a) $. Como deseja-se provar $\sim P(a)$, adiciona-se $ P(a) $ e a cláusula vazia é buscada. De P(a) e $ P(a) \implies Q(a) $, com o modus pones obtêm-se Q(a). Unindo as clausulas Q(a) e $ F_2: \sim Q(a) $ chega-se a cláusula vazia.

\underline{9.} Transform the following formulas into prenex normal forms:
\begin{itemize}
 \item[(1)] $ (\forall x) (P(x) \implies (\exists y) Q(x,y)) $ \newline
$(\forall x) (\exists y) P(x) \implies Q(x, y) $
 \item[(2)] $ (\exists x) (\sim((\exists y)P(x,y)) \implies ((\exists z) Q(z) \implies R(x))) $ \newline
$(\exists x)(\exists y)(\forall z) \sim P(x,y) \vee \sim Q(z) \vee R(x) $
 \item[(3)] $ (\forall x)(\forall y)((\exists z) P(x,y,x) \wedge((\exists u) Q(x,u) \implies (\exists v) Q(y,v))) $. \newline
$ (\forall x)(\forall y)(\exists z)(\forall u)(\exists v)(P(x,y,x) \wedge (\sim (Q(x,u) \vee Q(y,v)))) $
\end{itemize}
